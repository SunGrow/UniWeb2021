\documentclass[14pt]{extarticle}

\usepackage[utf8]{inputenc}
\usepackage[T2A,T1]{fontenc}
\usepackage[russian]{babel}
\usepackage[a4paper,lmargin=3cm,rmargin=2cm,tmargin=2cm,bmargin=2cm]{geometry}
\usepackage[ddmmyyyy]{datetime}
\usepackage{indentfirst}
\usepackage{hyperref}
\usepackage{graphicx}
\graphicspath{{./images/}}

\usepackage{minted}

\usepackage{titlesec}
\titleformat{\section}{\normalfont\bfseries\centering}{\thesection}{1em}{}
\titleformat{\subsection}{\normalfont\bfseries}{\thesubsection}{1em}{}
\usepackage{setspace}
\singlespacing
%\onehalfspacing
%\doublespacing
\setlength{\parindent}{1.25cm}
\let\oldsection\section
\renewcommand\section{\clearpage\oldsection}

\usepackage[backend=biber]{biblatex}
\addbibresource{~/texdoc/bibl.bib}
\input{~/texdoc/unibind.tex}
\usepackage{multirow}

\usepackage{pgf}
\usepackage{tikz}
\usetikzlibrary{arrows,automata}
\usetikzlibrary{positioning}

\tikzset{
	state/.style={
		rectangle,
		draw=black, very thick,
		anchor=west, align=left,
		text width=6cm,
	},
}
\usepackage{csquotes}
\usepackage{mathtools}

\begin{document}
\unititle
{\klgtu}
{\fapu}
{\suvt}
{ Лабораторная работа №5 \par Создание блога. }
{По дисциплине: «»}
{Доцент}
{}
{ст. гр. 18-ВТ}
{Поляков Л.Д.}

\tableofcontents

\section{Цель}

Изучить показатели качества, связанные с АСОИУ, познакомиться с общими способами организации обеспечения качества и стандартизации как основы обеспечения качества.

\section{Задание}

Для объекта: Автоматизированная система учета материальных средств выполнить следующие работы:
\begin{enumerate}

	\item Представьте три примера готовых программных продуктов, применяемых для Автоматизированных систем учета материальных средств;
	\item Приведите краткие характеристики приведенных программных продуктов;
	\item Сформулируйте единичные показатели качества и определите их меры. Результаты оформите в виде таблиц (формы таблиц представлены выше);
	\item Представьте кратко имеющиеся негативные отзывы о представленных программных продуктах
	\item Проанализируйте положительные и отрицательные стороны выбранных комплексов с точки зрения надежности. 
	\item Подготовьте и представьте отчет о проделанной работе соответствии с приведенным выше планом.
\end{enumerate}

\section{Ход работы}

\subsection{Название предметной области и ее программные продукты}

Предметная область: Автоматизированная система учета материальных средств

Программные продукты: Мой склад; CloudShop; УчетОблако; 1С:управление торговлей;

\subsection{Требования к техническому обеспечению системы}

\begin{tabular}{|c|c|c|c|c|c|}
	\hline
	Наименование комплекса & Микропроцессор & Объем ОЗУ & & & \\ \hline
	Мой склад & Не указаны & Не указаны & Не указаны & Не указаны & \\ \hline
	CloudShop & Не указаны & Не указаны & Не указаны & Не указаны & \\ \hline
	УчетОблако & Не указаны & Не указаны & Не указаны & Не указаны & \\ \hline
	1С:управление торговлей & Не указаны & Не указаны & Не указаны & Не указаны & \\ \hline
\end{tabular}

\subsection{Программные  характеристики системы}

\begin{tabular}{|c|c|c|c|c|c|}
	\hline
	Наименование комплекса & Требования к ОС & Обеспечение сетевой ОС &  & & \\ \hline
	Мой склад & Не указаны & Не указаны & Не указаны & Не указаны & \\ \hline
	CloudShop & Не указаны & Не указаны & Не указаны & Не указаны & \\ \hline
	УчетОблако & Не указаны & Не указаны & Не указаны & Не указаны & \\ \hline
	1С:управление торговлей & Не указаны & Не указаны & Не указаны & Не указаны & \\ \hline
\end{tabular}

\subsection{Функциональные  характеристики системы}

\begin{tabular}{|c|c|c|c|c|c|}
	\hline
	Наименование комплекса & Надежность & Устойчивость &  & & \\ \hline
	Мой склад &  &  &  & & \\ \hline
	CloudShop &  &  &  & & \\ \hline
	УчетОблако &  &  &  & & \\ \hline
	1С:управление торговлей &  &  &  & & \\ \hline
\end{tabular}

\subsection{Единицные показатели качества и их меры}

\begin{tabular}{|c|c|c|c|c|c|}
	\hline
	Наименование комплекса &  &  &  & & \\ \hline
	Мой склад &  &  &  & & \\ \hline
	CloudShop &  &  &  & & \\ \hline
	УчетОблако &  &  &  & & \\ \hline
	% Скриншот; В данном случае нарушены "Функциональные характеристики системы" (прим.); Связано это с ...;
	1С:управление торговлей &  &  &  & & \\ \hline
\end{tabular}




\end{document}
